\begin{abstract}

Rádios Cognitivos são dispositivos capazes de escutar o meio e escolher frequências disponíveis para a transmissão de dados. A escolha do canal depende da disponibilidade do mesmo. O rádio realiza uma consulta à um servidor , que retorna os canais que estão livres naquela localização. Diferentes modelos de propagação do sinal são implementados no servidor, para realização do cálculo dos canais livres. O trabalho consiste em desenvolver um servidor Web ilustrando o espectro no Estado do Rio de Janeiro, implementar um novo modelo de propagação que considere o relevo, garantir a persistência dos dados calculados anteriormente e permitir uma melhor interação do usuário com o servidor.

\end{abstract}


